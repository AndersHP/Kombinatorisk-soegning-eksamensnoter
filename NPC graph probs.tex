\documentclass{article}
\usepackage[utf8]{inputenc}

\title{NP-Complete graph problems - KombSøg 2017}
        \author{Anders H Pedersen }
        \date{May 2017}
\begin{document}

\maketitle

\begin{center}
\textit{The graph problems and reductions below were introduced by Thomas Dueholm Hansen, at the lecture on NP-Complete graph problems may 3 2017.}
\end{center}

\section{Graph problems}
\subsection{INDEPENDENT SET}
\textbf{Given:} Graph G = (V,E), target K. \\
\textbf{Question:} Does there exist $ \subseteq$ \\
\subsection{CLIQUE}
\textbf{Given:} \\
\textbf{Question:} \\
\subsection{VERTEX COVER}
\textbf{Given:} \\
\textbf{Question:} \\
\subsection{MAX CUT}
\textbf{Given:} \\
\textbf{Question:} \\
\subsection{BISECTION}
\textbf{Given:} \\
\textbf{Question:} \\
\subsection{HAMILTONIAN PATH}
\textbf{Given:} \\
\textbf{Question:} \\
\subsection{TSP}
\textbf{Given:} \\
\textbf{Question:} \\
\section{Reductions}
As always, in a reduction $L_1 \le L_2$, we describe a polynomial time computable function r, such that $\forall x: x\in L_1 \iff r(x) \in L_2$. We then argue that it is indeed polynomial, and that both directions of the bi-implication holds.
\subsection{3SAT $\le$ INDEPENDENT SET}
We prove \textbf{Theorem 9.4: INDEPENDENT SET is NP-Complete}, by reducing from 3SAT, which we know is NP-Complete.\\\\
For this reduction we need a gadget, the triangle. The logic behind this is that, if a graph contains a triangle, then at most one of the nodes can be in the independent set. We restrict the class of graphs we consider, to graphs whose nodes can be partitioned in m disjoint triangles. 
\\\\
Read more in papadimitriou page 188-190.
\subsection{INDEPENDENT SET $\le$ CLIQUE}
\subsection{INDEPENDENT SET $\le$ VERTEX COVER}
\subsection{NAESAT $\le$ MAX CUT}
\subsection{3SAT $\le$ HAMILTONIAN PATH}
\subsection{HAMILTONIAN PATH $\le$ TSP}



\end{document}
